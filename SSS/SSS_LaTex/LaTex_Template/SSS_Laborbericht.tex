%---------------
%╔═╗╔═╗╔╦╗╦ ╦╔═╗
%╚═╗║╣  ║ ║ ║╠═╝
%╚═╝╚═╝ ╩ ╚═╝╩  
%---------------

% language setup
\newcommand{\docLanguage}{ngerman}
%\newcommand{\docLanguage}{english}

% DOCUMENT SETUP
\documentclass[12pt, oneside, a4paper, \docLanguage]{report}
\usepackage[left=3cm, 
			right=2.5cm, 
			top=2.5cm, 
			bottom=2.5cm, 
			includehead, 
			includefoot]{geometry}

% line spacing
\usepackage{setspace}
\setstretch{1,25} % 15/12 --> 1.25

% encoding setup
% T1 font encoding for languages that use a latin alphabet
\usepackage[T1]{fontenc} 

% enhanced input encoding handling - utf8 for äÄüÜöÖß...
\usepackage[utf8]{inputenc}

%de­fines Adobe Times Ro­man as de­fault text font
\usepackage{mathptmx}
\usepackage{times} % needed for acronym package

%PDF linking package
\usepackage[hidelinks]{hyperref}


% Language Setup
\usepackage[\docLanguage]{babel}
% after babel - set chapter string
\AtBeginDocument{\renewcommand{\chaptername}{}}

% language specific bibliography style
\usepackage[numbers, square]{natbib}
%\setcitestyle{square,aysep={},yysep={;}}
\usepackage[fixlanguage]{babelbib}
\selectbiblanguage{\docLanguage}
% bliographystyle setup
% babel specific: babplain, babplai3, babalpha, babunsrt, bababbrv, bababbr3
\bibliographystyle{babunsrt}


% enumeration
\usepackage{enumitem}
% tabular extension tabularx
\usepackage{tabularx}

% math packages
\usepackage{amsmath}
\usepackage{nicefrac}
\usepackage{amsthm}
\usepackage{amsbsy}
\usepackage{amssymb}
\usepackage{amsfonts}
%\usepackage{MnSymbol}


%special characters
\usepackage{amssymb}
\usepackage{upgreek,textgreek}

% acronym package
\usepackage[printonlyused, footnote]{acronym}

% breakable text in \seqsplit{}
\usepackage{seqsplit}

% \textmu
\usepackage{textcomp}

% package provides a way to compile sections of a document using the same preamble as the main document
\usepackage{subfiles}

% driver-independent color extension - used by listings,tabularx
\usepackage[usenames,dvipsnames,table,xcdraw]{xcolor}

% -- SYNTAX HIGHLIGHTING --
\usepackage{listings}
%% bash command line Syntax Highlighting
\lstdefinestyle{BASH_CMD}{ 
  columns=fullflexible,            % copy pasteable listings
  language=bash,
  basicstyle=\small\sffamily,
  basicstyle   = \small \ttfamily,
  keywordstyle = [1]\small \ttfamily,
  keywordstyle = [2]\small \ttfamily,
  commentstyle = \small \ttfamily,
  numbers=none,
  captionpos=b, 
  breaklines=true,
  numberstyle=\tiny,
  numbersep=3pt,
  frame=tlrb,
  columns=fullflexible,
  backgroundcolor=\color{white!20},
  linewidth=\linewidth,
  literate=                        % replace in code
     {Ö}{{\"O}}1 
     {Ä}{{\"A}}1 
     {Ü}{{\"U}}1 
     {ß}{{\ss}}2 
     {ü}{{\"u}}1 
     {ä}{{\"a}}1 
     {ö}{{\"o}}1 
     {â}{{\^{a}}}1 
     {Â}{{\^{A}}}1 
     {ç}{{\c{c}}}1 
     {Ç}{{\c{C}}}1 
     {ğ}{{\u{g}}}1 
     {Ğ}{{\u{G}}}1 
     {ı}{{\i}}1 
     {İ}{{\.{I}}}1 
     {ş}{{\c{s}}}1 
     {Ş}{{\c{S}}}1 
}
 % adds style BASH_CMD
%\input{cfgs/listings/listings_def_lang_bash-script.tex} % adds style BASH_SCRIPT
\input{cfgs/listings/listings_def_lang_latex.tex} % adds style LATEX
%\input{cfgs/listings/listings_def_lang_matlab.tex} % adds style MATLAB
\input{cfgs/listings/listings_def_lang_python.tex} % adds style PYTHON
%% Matlab Syntax Highlighting
\colorlet{keyword}{blue!100!black!80}
\colorlet{STD}{Lavender}
\colorlet{comment}{green!90!black!90}
\definecolor{mygreen}{rgb}{0,0.6,0}
\definecolor{mygray}{rgb}{0.5,0.5,0.5}
\definecolor{mymauve}{rgb}{0.58,0,0.82}


\lstdefinestyle{CPP}{ 
  language     = C++,
  basicstyle   = \footnotesize \ttfamily,
  keywordstyle = [1]\color{keyword}\bfseries,
  keywordstyle = [2]\color{STD}\bfseries,
  commentstyle = \color{mygreen}\itshape,
  backgroundcolor=\color{white},   % choose the background color; you must add \usepackage{color} 
                                   % or \usepackage{xcolor}
  columns=fullflexible,            % copy pasteable listings
  basicstyle=\footnotesize,        % the size of the fonts that are used for the code
  breakatwhitespace=false,         % sets if automatic breaks should only happen at whitespace
  breaklines=false,                % sets automatic line breaking
  captionpos=c,                    % sets the caption-position to bottom
  extendedchars=true,              % lets you use non-ASCII characters; for 8-bits encodings only,
                                   % does not work with UTF-8
  frame=single,                    % adds a frame around the code
  keepspaces=true,                 % keeps spaces in text, useful for keeping indentation of code
                                   % (possibly needs columns=flexible)
  numbers=left,                    % where to put the line-numbers; possible values are 
                                   % (none, left, right)
  numbersep=5pt,                   % how far the line-numbers are from the code
  numberstyle=\tiny\color{mygray}, % the style that is used for the line-numbers
  rulecolor=\color{black},         % if not set, the frame-color may be changed on line-breaks
                                   % within not-black text (e.g. comments (green here))
  showspaces=false,                % show spaces everywhere adding particular underscores; it
  	                               % overrides 'showstringspaces'
  showstringspaces=false,          % underline spaces within strings only
  showtabs=false,                  % show tabs within strings adding particular underscores
  stepnumber=1,                    % the step between two line-numbers. If it's 1, each line 
                                   % will be numbered
  stringstyle=\color{mymauve},     % string literal style
  tabsize=2,                       % sets default tabsize to 2 spaces
  title=\lstname,                  % set title name
  literate=                        % replace in code
     {Ö}{{\"O}}1 
     {Ä}{{\"A}}1 
     {Ü}{{\"U}}1 
     {ß}{{\ss}}2 
     {ü}{{\"u}}1 
     {ä}{{\"a}}1 
     {ö}{{\"o}}1 
     {â}{{\^{a}}}1 
     {Â}{{\^{A}}}1 
     {ç}{{\c{c}}}1 
     {Ç}{{\c{C}}}1 
     {ğ}{{\u{g}}}1 
     {Ğ}{{\u{G}}}1 
     {ı}{{\i}}1 
     {İ}{{\.{I}}}1 
     {ş}{{\c{s}}}1 
     {Ş}{{\c{S}}}1 
} % adds style CPP
%\input{cfgs/listings/listings_def_lang_c.tex} % adds style C
%\input{cfgs/listings/listings_def_lang_json.tex} % adds style JSON

% HEADLINE CFG
\usepackage{fancyhdr} % Headers and footers
\usepackage{lastpage}
\usepackage{ifthen}
\setlength{\headheight}{1.5cm}
%\pagestyle{fancy} % All pages have headers and footers
% override plain page style for \part, \chapter or 
% \maketitle, which implicit specifies plain page style
\fancypagestyle{plain} 
{
	\fancyhead[L]{}
	\fancyhead[C]{}
	\fancyhead[R]{}
	\fancyfoot[L]{}
	\fancyfoot[C]{\thepage}
	\fancyfoot[R]{}
}
% set list pagestyle
\fancypagestyle{preface} 
{
	\fancyhead[L]{}
	\fancyhead[C]{}
	\fancyhead[R]{}
	\fancyfoot[L]{}
	\fancyfoot[C]{\thepage}
	\fancyfoot[R]{}
}
% set default pagestyle
\fancypagestyle{default} 
{
	\fancyhead{} % Blank out the default header
	\fancyfoot{} % Blank out the default footer
	\fancyhead[L]{}
	\fancyhead[C]{}
	\fancyhead[R]{}
	\fancyfoot[L]{}
	\fancyfoot[C]{\thepage}
	\fancyfoot[R]{}
}
%\fancypagestyle{default} 
{
\fancyhead[L]{\ifthenelse{\isodd{\value{page}}}{\arabic{chapter} \rightmark}{}}
\fancyhead[R]{\thepage}
}

\renewcommand{\chaptermark}[1]{\markright{#1}{}}
\renewcommand{\sectionmark}[1]{\markright{#1}{}}
\renewcommand{\headrulewidth}{0pt}
\renewcommand{\footrulewidth}{0pt}

% PICTURE CFG 
\usepackage{verbatim}
\usepackage{graphicx}
\usepackage{epstopdf}
\usepackage{caption}
\usepackage[list=true,listformat=simple]{subcaption}
% floating prevention packages
\usepackage{float}    % used with [H] positioning parameter
\usepackage{placeins} % \FloatBarrier 
% tikz packages
\usepackage{tikz}
\usepackage{standalone}
\usepackage{pgfplots}


% include only specified tex files - uncommend here
\includeonly{preface/cover,
             preface/abstract,
             preface/tableofcontents,
             preface/listoffigures,
             preface/listoftables,
             preface/lstlistoflistings,
             appendix/bibliography}

%-------------------
%╔═╗╔╦╗╦═╗╦╔╗╔╔═╗╔═╗
%╚═╗ ║ ╠╦╝║║║║║ ╦╚═╗
%╚═╝ ╩ ╩╚═╩╝╚╝╚═╝╚═╝
%-------------------
\newcommand{\strLecture}{Signale, Systeme und Sensoren}
\newcommand{\strDate}{\today}
\newcommand{\strAuthorA}{A. Author}
\newcommand{\strAuthorB}{B. Author}
%\newcommand{\strAuthorC}{C. Author}
\newcommand{\strAuthorAEmail}{aauthor@htwg-konstanz.de}
\newcommand{\strAuthorBEmail}{bauthor@htwg-konstanz.de}
%\newcommand{\strAuthorCEmail}{cauthor@htwg-konstanz.de}
% Versuchsbeschreibung 
\newcommand{\strTopic}{VERSUCH NAME}
\newcommand{\strAbstract}{Zusammenfassung etwa 100 Worte.}
% hyperref customization
\hypersetup{
	pdftitle     = {\strTopic}, % title
	pdfsubject   = {\strLecture}, % subject of the document
	pdfauthor    = {\strAuthorA, \strAuthorB}, % author
	pdfkeywords  = {}, % list of keywords
	pdfcreator   = {}, % creator of the document
	pdfproducer  = {}, % producer of the document
	colorlinks   = false, % false: boxed links; true: colored links
	linkcolor    = red, % color of internal links (change box color with linkbordercolor)
    citecolor    = green, % color of links to bibliography
    filecolor    = magenta, % color of file links
    urlcolor     = cyan, % color of external links
	%bookmarks    = true, % show bookmarks bar?
	unicode	     = true, % non-Latin characters in Acrobat’s bookmarks
	pdftoolbar   = true, % show Acrobat’s toolbar?
	pdfmenubar   = true, % show Acrobat’s menu?
    pdffitwindow = false, % window fit to page when opened
	pdfnewwindow = true % links in new PDF window
}

%-----------------------------------------
% ╔╗ ╔═╗╔═╗╦╔╗╔  ╔╦╗╔═╗╔═╗╦ ╦╔╦╗╔═╗╔╗╔╔╦╗ 
% ╠╩╗║╣ ║ ╦║║║║   ║║║ ║║  ║ ║║║║║╣ ║║║ ║  
% ╚═╝╚═╝╚═╝╩╝╚╝  ═╩╝╚═╝╚═╝╚═╝╩ ╩╚═╝╝╚╝ ╩  
%-----------------------------------------

\begin{document}
\pagenumbering{Roman} 

\setcounter{section}{0}

\begin{titlepage}

\vspace*{-3.5cm}

\begin{flushleft}
\hspace*{-1cm} \includegraphics[width=15.7cm]{preface/htwg-logo}
\end{flushleft}

\vspace{1cm}

\begin{center}
	\large{
		\textbf{\strLecture} \\[2cm]
	}
	\Huge{
		\textbf{\strTopic} \\[2cm]
	}
	\Large{
		\textbf{\strAuthorA, \strAuthorB}} \\[3cm]
		%\textbf{\strAuthorA, \strAuthorB, \strAuthorC}} \\[3cm]
	\large{
		\textbf{} \\[2.3cm]
	}
	
	\large{
		\textbf{Konstanz, \strDate}
	}
\end{center}

\end{titlepage}
\thispagestyle{empty}




\begin{center}
{\Large \textbf{Zusammenfassung (Abstract)}}
\end{center}

\bigskip

\begin{center}
	\begin{tabular}{p{2.8cm}p{5cm}p{5cm}}
		Thema: & \multicolumn{2}{p{10cm}}{\raggedright\strTopic} \\
		 & & \\
		Autoren: & \strAuthorA & \href{mailto:\strAuthorAEmail}{\strAuthorAEmail} \\
		 & \strAuthorB & \href{mailto:\strAuthorBEmail}{\strAuthorBEmail} \\
%		 & \strAuthorC & \href{mailto:\strAuthorCEmail}{\strAuthorCEmail} \\
		 & & \\
		Betreuer: & Prof. Dr. Matthias O. Franz & \href{mailto:mfranz@htwg-konstanz.de}{mfranz@htwg-konstanz.de} \\
		 &  Jürgen Keppler & \href{mailto:juergen.keppler@htwg-konstanz.de}{juergen.keppler@htwg-konstanz.de} \\
		 &  Mert Zeybek & \href{mailto:me431zey@htwg-konstanz.de}{me431zey@htwg-konstanz.de} \\
	\end{tabular}
\end{center}

\bigskip

\noindent
\strAbstract

\thispagestyle{preface}



\clearpage

%
% TABLE OF CONTENTS
%
\pagestyle{preface}
%
% TABLE OF CONTENTS
%
\tableofcontents
\newpage


%
% Abbildungsverzeichnis
%
%
% Abbildungsverzeichnis
%
\phantomsection
\addcontentsline{toc}{chapter}{Abbildungsverzeichnis}
\listoffigures
\thispagestyle{preface}
\newpage
\clearpage

%
% Tabellenverzeichnis
%
%
% Tabellenverzeichnis
%
\phantomsection
\addcontentsline{toc}{chapter}{Tabellenverzeichnis}
\listoftables
\thispagestyle{preface}
\newpage
\clearpage

%
% Listingverzeichnis
%
%
% Listingverzeichnis
%
\phantomsection
\renewcommand\lstlistingname{Listing}
\renewcommand\lstlistlistingname{Listingverzeichnis}
\lstlistoflistings
\addcontentsline{toc}{chapter}{Listingverzeichnis}
\thispagestyle{preface}
\newpage
\clearpage


%--------------------------
% ╔═╗╦ ╦╔═╗╔═╗╔╦╗╔═╗╦═╗╔═╗ 
% ║  ╠═╣╠═╣╠═╝ ║ ║╣ ╠╦╝╚═╗ 
% ╚═╝╩ ╩╩ ╩╩   ╩ ╚═╝╩╚═╚═╝ 
%--------------------------

\pagenumbering{arabic} 
\setcounter{page}{1} 
\pagestyle{default}
%
% CHAPTER Einleitung
%
\chapter{Einleitung}
\label{chap:EINL}

\cite{Franz2016n}
\cite{Franz2016e}

%
% CHAPTER Versuch 1
%
\chapter{Versuch 1}
\label{chap:VERSUCH_1}

\section{Fragestellung, Messprinzip, Aufbau, Messmittel}
\label{chap:VERSUCH_1_FRAGESTELLUNG}

\section{Messwerte}
\label{chap:VERSUCH_1_MESSWERTE}

\section{Auswertung}
\label{chap:VERSUCH_1_AUSWERTUNG}

\section{Interpretation}
\label{chap:VERSUCH_1_INTERPRETATION}

%
% CHAPTER Versuch 2
%
\chapter{Versuch 2}
\label{chap:VERSUCH_2}

\section{Fragestellung, Messprinzip, Aufbau, Messmittel}
\label{chap:VERSUCH_2_FRAGESTELLUNG}

\section{Messwerte}
\label{chap:VERSUCH_2_MESSWERTE}

\section{Auswertung}
\label{chap:VERSUCH_2_AUSWERTUNG}

\section{Interpretation}
\label{chap:VERSUCH_2_INTERPRETATION}

%
% CHAPTER Versuch 3
%
\chapter{Versuch 3}
\label{chap:VERSUCH_3}

\section{Fragestellung, Messprinzip, Aufbau, Messmittel}
\label{chap:VERSUCH_3_FRAGESTELLUNG}

\section{Messwerte}
\label{chap:VERSUCH_3_MESSWERTE}

\section{Auswertung}
\label{chap:VERSUCH_3_AUSWERTUNG}

\section{Interpretation}
\label{chap:VERSUCH_3_INTERPRETATION}

%
% CHAPTER Versuch 4
%
\chapter{Versuch 4}
\label{chap:VERSUCH_4}

\section{Fragestellung, Messprinzip, Aufbau, Messmittel}
\label{chap:VERSUCH_4_FRAGESTELLUNG}

\section{Messwerte}
\label{chap:VERSUCH_4_MESSWERTE}

\section{Auswertung}
\label{chap:VERSUCH_4_AUSWERTUNG}

\section{Interpretation}
\label{chap:VERSUCH_4_INTERPRETATION}
%
% CHAPTER Anhang
%
\renewcommand\thesection{A.\arabic{section}}
\renewcommand\thesubsection{\thesection.\arabic{subsection}}

\chapter*{Anhang}
\label{chap:APPENDIX}
\addcontentsline{toc}{chapter}{Anhang}
%\setcounter{chapter}{0}
\addtocounter{chapter}{1}
\setcounter{section}{0}

\section{Quellcode}
\label{chap:APPENDIX_SOURCECODE}

\subsection{Quellcode Versuch 1}
\label{chap:APPENDIX_SOURCECODE_V1}

\subsection{Quellcode Versuch 2}
\label{chap:APPENDIX_SOURCECODE_V2}

\subsection{Quellcode Versuch 3}
\label{chap:APPENDIX_SOURCECODE_V3}

\subsection{Quellcode Versuch 4}
\label{chap:APPENDIX_SOURCECODE_V4}


\section{Messergebnisse}
\label{chap:APPENDIX_MEASUREMENT_SOURCE}

%
% Literaturverzeichnis
%
%
% Literaturverzeichnis
%
\phantomsection
\addcontentsline{toc}{chapter}{Literaturverzeichnis}
\bibliography{../references}
\newpage

\end{document}
%------------------------------------
% ╔═╗╔╗╔╔╦╗  ╔╦╗╔═╗╔═╗╦ ╦╔╦╗╔═╗╔╗╔╔╦╗
% ║╣ ║║║ ║║   ║║║ ║║  ║ ║║║║║╣ ║║║ ║ 
% ╚═╝╝╚╝═╩╝  ═╩╝╚═╝╚═╝╚═╝╩ ╩╚═╝╝╚╝ ╩ 
%------------------------------------